% !TEX program = pdflatex
\documentclass{article}

\usepackage{blindtext}

\usepackage[]{geometry}

\title{Length Commands}
\author{naiithink}
\date{}

\setlength{\parindent}{0pt}

\begin{document}
\maketitle
\tableofcontents
\newpage

\section{Setting Values of Length Commands}

\begin{verbatim}
\newlength              Defines a new length command
\setlength
\addtolength
\settowidth
\settoheight
\settodepth
\end{verbatim}

\section{\texttt{parindent}}

\verb|\parindent|
\newpage


\section{Baselines}

% Save default value
\let\defaultbaselineskip\baselineskip
Defaults to \the\defaultbaselineskip

\subsection{\texttt{baselineskip}}

\subsubsection{baselineskip: 12pt}

\setlength{\baselineskip}{12pt}
\blindtext[1]

\subsubsection{baselineskip: 14pt}

\setlength{\baselineskip}{14pt}
\blindtext[1]

\subsubsection{baselineskip: 16pt}

\setlength{\baselineskip}{16pt}
\blindtext[1]

% Reset to default
\setlength{\baselineskip}{\defaultbaselineskip}


\subsection{\texttt{baselinestretch}}
Multiplies the value of \verb|\baselineskip| --- applies to the entire document, including footnotes and tables ---
In principle, ``double spacing'' can be obtained by \verb|\renewcommand{\baselinestretch}{2}|.
\newpage

\section{\texttt{parskip}}
Extra vertical space inserted before a paragraph.

\let\defaultparskip\parskip
Defaults to: \the\defaultparskip

\blindtext[1]

\setlength{\parskip}{16pt}
\blindtext[1]

\setlength{\parskip}{32pt}
\blindtext[1]

\setlength{\parskip}{\defaultparskip}
\end{document}
