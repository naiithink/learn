\documentclass{article}

\usepackage{geometry}
\geometry{a4paper, portrait, margin=1in}
\usepackage[thaifont = THSarabunNew]{thaispec}
\usepackage[linguistics]{forest}

\title{Assignment IV, Basic Counting}
\author{พศวัต ถิ่นกาญจน์วัฒนา}
\date{}

\begin{document}

\maketitle
\flushleft
\renewcommand{\labelenumii}{\arabic{enumii})}

\begin{enumerate}
	\item จงหาจำนวนวิธีทั้งหมดในการชนะ 3 เกมจากทั้งหมด 5 เกม ระหว่างผู้เล่น 2 คน คือ $T_1$ และ $T_2$\newline
	และอยากทราบว่าผู้เล่นทั้ง 2 คน จะสามารถชนะเกมได้ทีมละกี่เกม (อาจใช้ Tree diagram)\newline
	\begin{forest}
	for tree={grow=0, s sep=0}
		[เกม
			[$W$
				[$W$
					[$W$
						[$W$
							[$W$]
							[$L$]
						]
						[$L$
							[$W$]
							[$L$]
						]
					]
					[$L$
						[$W$
							[$W$]
							[$L$]
						]
						[$L$
							[$W$]
							[$L$]
						]
					]
				]
				[$L$
					[$W$
						[$W$
							[$W$]
							[$L$]
						]
						[$L$
							[$W$]
							[$L$]
						]
					]
					[$L$
						[$W$
							[$W$]
							[$L$]
						]
						[$L$
							[$W$]
							[$L$]
						]
					]
				]
			]
			[$L$
				[$W$
					[$W$
						[$W$
							[$W$]
							[$L$]
						]
						[$L$
							[$W$]
							[$L$]
						]
					]
					[$L$
						[$W$
							[$W$]
							[$L$]
						]
						[$L$
							[$W$]
							[$L$]
						]
					]
				]
				[$L$
					[$W$
						[$W$
							[$W$]
							[$L$]
						]
						[$L$
							[$W$]
							[$L$]
						]
					]
					[$L$
						[$W$
							[$W$]
							[$L$]
						]
						[$L$
							[$W$]
							[$L$]
						]
					]
				]
			]
		]
	\end{forest}
	\item จงหาจำนวนวิธีทั้งหมดในการจัดที่นั่งให้คน 4 คนในโต๊ะกลม โดยการจัดที่นั่งที่คนด้านซ้ายและขวาเป็นคนเดิม\newline
	ถือว่าเป็นการจัดที่นั่งซ้ำ
	\item จงหาจำนวนสมาชิกของ $|A \cup B|$ เมื่อ $|A| = 12$ และ $|B| = 18$ เมื่อแต่ละกรณีต่อไปนี้เป็นจริง
	\begin{enumerate}
		\item $|A \cap B| = \phi$
		\item $|A \cap B| = 1$
		\item $|A \cap B| = 6$
		\item $A \subseteq B$
	\end{enumerate}
	\item อักขระภาษาอังกฤษประกอบด้วยพยัญชนะ 21 ตัว สระ 5 ตัว จงหาสตริงของอักขระตัวเล็ก (lowercase)\newline
	ความยาว 6 ที่ประกอบด้วย
	\begin{enumerate}
		\item มีสระ 1 ตัว
		\item มีสระ 2 ตัว โดยสามารถเลือกสระซ้ำได้
		\item มีสระอย่างน้อย 1 ตัว
		\item มีสระอย่างน้อย 2 ตัว
	\end{enumerate}
	\item การจับสลากของเลข 1 ถึง 100 เพื่อมอบรางวัลจำนวน 4 รางวัล ประกอบด้วยรางวัลที่ 1, 2, 3 และรางวัลพิเศษ\newline
	จงหาจำนวนวิธีในการมอบรางวัลทั้งสี่ภายใต้เงื่อนไข ดังนี้
	\begin{enumerate}
		\item ไม่มีกติกาเพิ่มเติม
		\item ผู้ที่ถือสลากหมายเลข 47 ได้รับรางวัลพิเศษ
		\item ผู้ที่ถือสลากหมายเลข 47 ได้รับรางวัลใดรางวัลหนึ่งในสี่รางวัล
		\item ผู้ที่ถือสลากหมายเลข 19, 47, 73 หรือ 97 เป็นหมายเลขของรางวัลทั้งสี่
	\end{enumerate}
	\item จงหาจำนวนเต็มบวกมีค่าไม่เกิน 1,000 ที่หารด้วย 7 หรือ 11 ลงตัว
	\item จงหาจำนวนวิธีเลือกไพ่ 5 ใบ โดยมีไพ่อย่างน้อย 1 ใบจากแต่ละ suit
	\item กำหนดให้ไพ่หนึ่งสำรับมี 52 ใบ จงหาจำนวนวิธีในการแจกไพ่จำนวน 5 ใบให้กับผู้เล่น 4 คน
	\item จงหาจำนวนวิธีในการหยิบลูกบอลลักษณะเดียวกันจำนวน 10 ลูกใส่ในตะกร้า 8 ตะกร้าที่มีหมายเลข 1 ถึง 8\newline
	ติดอยู่ที่ด้านข้างตะกร้าแต่ละใบ
	\item จงหาจำนวนคำตอบที่เป็นไปได้ของสมการ $x_1 + x_2 + \dots + x_n = k$ เมื่อ $x_i \geq 0$

	\pagebreak

	\item จาก Binomial theorem จงกระจายพจน์ทั้งหมดของ
	\begin{enumerate}
		\item ${(x + 1)}^4$
		\item ${(2 + y)}^4$
	\end{enumerate}
	\item โยนเหรียญจำนวน 10 ครั้ง แต่ละครั้งออกหัวหรือก้อยด้วยความน่าจะเป็นที่เท่ากัน ให้หาค่าต่อไปนี้
	\begin{enumerate}
		\item จำนวน outcomes ที่เป็นไปได้ทั้งหมด
		\item จำนวน outcomes ที่ออกหัวจำนวน 2 ครั้ง
		\item จำนวน outcomes ที่ออกก้อยอย่างมาก 3 ครั้ง
		\item จำนวน outcomes ที่ออกหัวและก้อยในจำนวนที่เท่ากัน
	\end{enumerate}
\end{enumerate}

\end{document}