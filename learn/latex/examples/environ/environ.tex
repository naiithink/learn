% !TEX program = xelatex
\documentclass[a4paper]{article}

\usepackage{blindtext}

\title{Environments}
\author{naiithink}
\date{}

\begin{document}
\begin{titlepage}
\maketitle
\thispagestyle{empty}
\tableofcontents
\end{titlepage}

\newpage
\section{Lists}
\flushleft
\begin{enumerate}
\item You can nest the list
environments to your taste:
\begin{itemize}
\item But it might start to
look silly.
\item[-] With a dash.
\end{itemize}
\item Therefore remember:
\begin{description}
\item[Stupid] things will not
become smart because they are
in a list.
\item[Smart] things, though,
can be presented beautifully
in a list.
\end{description}
\end{enumerate}

\newpage
\section{Alignment}
\begin{flushleft}
\blindtext
\end{flushleft}

\begin{center}
\blindtext
\end{center}

\begin{flushright}
\blindtext
\end{flushright}

\newpage
\section{Quotes}

\subsection{\texttt{quote}, \texttt{quotation}}
\subsubsection*{\texttt{quote}}
A typographical rule of thumb
for the line length is:
\begin{quote}
On average, no line should
be longer than 66 characters.
\end{quote}
This is why \LaTeX{} pages have
such large borders by default
and also why multicolumn print
is used in newspapers.

\subsubsection*{\texttt{quotation}}
The quotation environment is useful for longer quotes
going over several paragraphs, because it indents the first
line of each paragraph.

\subsection{\texttt{verse}}
I know only one English poem by
heart. It is about Humpty Dumpty.
\begin{flushleft}
\begin{verse}
Humpty Dumpty sat on a wall:\\
Humpty Dumpty had a great fall.\\
All the King's horses and all
the King's men\\
Couldn't put Humpty together
again.
\end{verse}
\end{flushleft}

\newpage
\section{Abstract}
\begin{abstract}
\blindtext
\end{abstract}

\newpage
\section{Printing Verbatim}
The verbatim environment and the \texttt{\char92 verb} command may not be used within parameters of other commands.

\subsection{As a Block}
\begin{verbatim}
#include <stdio.h>

int main(void)
{
    printf("hello, world\n");
    return 0;
}
\end{verbatim}

\subsubsection*{The \texttt{*} version}
\begin{verbatim*}
#include <stdio.h>

int main(void)
{
    printf("hello, world\n");
    return 0;
}
\end{verbatim*}

\subsection{Interpolate}
Or within a \verb+paragraph+ eiei.\\
The \verb|+| is just an example of a delimiter character.\\
Use any character except letters, \verb|*| or space. Many \LaTeX{} examples in this booklet are typeset with this command.

\subsubsection*{Also has a \texttt{*} version}
\verb*|like this :-) |

\newpage
\section{Tabulation}
\begin{verbatim}
\begin{tabular}[pos]{table spec}
\end{verbatim}

\subsection*{\texttt{table spec}}
\begin{tabular}{c @{~~~~} l}
\texttt{l} & column of left-aligned text\\
\texttt{r} & right-aligned text\\
\texttt{c} & centered text\\
\texttt{p\{width\}} & column containing justified text with line breaks\\
\texttt{\textbar} & vertical line
\end{tabular}
\newline \newline

If the text in a column is too wide for the page, \LaTeX{} won't automatically wrap it.
Using \verb+p{width}+ you can define a special type of column which will wrap-around the text as in a normal paragraph.\\

\subsection*{Simple Table}
\begin{tabular}{|r|l|}
\hline
7C0 & hexadecimal \\
3700 & octal \\ \cline{2-2}
11111000000 & binary \\
\hline \hline
1984 & decimal \\
\hline
\end{tabular}

\subsection*{Wrapping}
\begin{tabular}{|p{4.7cm}|}
\hline
Welcome to Boxy's paragraph.
We sincerely hope you'll
all enjoy the show.\\
\hline
\end{tabular}

\subsection*{Leading Spaces}
\begin{tabular}{l | l}
\hline \\
\begin{tabular}{@{} l @{}}
\hline
no leading space\\
\hline
\end{tabular} &
\begin{tabular}{l}
\hline
leading space left and right\\
\hline
\end{tabular}\\\\
\hline
\end{tabular}

\subsection*{Decimal Point}
\begin{tabular}{c r @{.} l}
Pi expression       &
\multicolumn{2}{c}{Value} \\
\hline
$\pi$               & 3&1416  \\
$\pi^{\pi}$         & 36&46   \\
$(\pi^{\pi})^{\pi}$ & 80662&7 \\
\end{tabular}

\subsection*{Column Labeling}
\begin{tabular}{l | l}
\hline \\
\begin{tabular}{c r @{.} l}
Pi expression       &
\multicolumn{2}{c}{Value} \\
\hline
$\pi$               & 3&1416  \\
$\pi^{\pi}$         & 36&46   \\
$(\pi^{\pi})^{\pi}$ & 80662&7 \\
\end{tabular} &
\begin{tabular}{|c|c|}
\hline
\multicolumn{2}{|c|}{Ene} \\
\hline
Mene & Muh! \\
\hline
\end{tabular}\\
\hline
\end{tabular}

\newpage
\subsection{Long Tables}
% \begin{longtable}{| l | l |}
% \hline
% \blindtext[5] & \blindtext[5]\\
% \hline
% \end{longtable}

\end{document}