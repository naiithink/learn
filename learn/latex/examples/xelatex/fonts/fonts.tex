% !TEX program = xelatex
\documentclass{article}

\usepackage{geometry}
\usepackage{calc}
\usepackage{blindtext}
\usepackage[utf8]{inputenc}
\usepackage{fontspec}
\usepackage{polyglossia}
\usepackage{fancyhdr}
% \usepackage{hyperref}

\geometry{%
a4paper,
margin = 1in}

% \geometry{%
% a5paper,
% margin = 0.5in}

\renewcommand{\sectionmark}[1]{%
\markright{#1}}
\renewcommand{\subsectionmark}[1]{%
\markboth{#1}{}}

%\setmainfont[SizeFeatures = {Size=16}]{TH Sarabun New}
\setmainfont{Sarabun}

% \setmainfont{test}[
%     Extension=.ttf,
%     Path=src/,
%     UprightFont=test]

\setdefaultlanguage[numerals=arabic]{thai}
\setotherlanguages{english}

\title{\flushleft\Huge\textbf{เคล็ดลับความสุข จากผู้สูงอายุ 1,000 คน}\\
\normalsize\rule{0em}{1ex}\\
\Large Single Being EP.120}
\author{}
\date{}

\begin{document}
\maketitle
\sloppy
% \flushleft

ในสมัยนี้ เส้นแบ่งระหว่างวัยกลางคนกับวัยสูงอายุเริ่มจางลง สังเกตได้จากคนที่อายุ 60---70 ปีที่ดูภายนอกแล้วเหมือน
กับว่าอายุเพียงแค่ 40---50 ปี หรือที่เราเรียกว่า \mbox{``Young Old''} ซึ่งสิ่งหนึ่งที่ทำให้เส้นแบ่งนี้จางลงได้คือ องค์ความรู้
ทางการแพทย์ที่ก้าวหน้ามากขึ้น

\rule{0em}{1ex}

Dr. Karl Pillemer อาจารย์จาก Cornell University ได้ทำการศึกษาเกี่ยวกับเคล็ดลับในการมีความสุขกับชีวิต
ในมุมมองของผู้สูงอายุ โดยอาจารย์ฯ ได้เริ่มการศึกษาโดยให้ผู้สูงอายุจำนวนประมาณ 100 คน ได้เขียนเล่าแบบ
free-style ทำนองว่าอะไรคือกุญแจสำคัญของความสุข และหากย้อนเวลากลับไปได้อยากจะบอกอะไรบ้าง
หลังจากนั้น อาจารย์ฯ ก็ได้ทำแบบสอบถามส่งให้กับผู้สูงอายุอีกหลายร้อยคน ทำให้ได้ข้อมูลที่น่าสนใจชุดหนึ่ง
ที่บางคำตอบมีมุมมองที่น่าสนใจ อาจารย์ฯ จึงได้เชิญผู้สูงอายุเหล่านั้นมาสัมภาษณ์เชิงลึกเพื่อเจาะลึกถึงข้อมูล

\rule{0em}{1ex}

รวม ๆ แล้ว การศึกษานี้มีผู้ที่อายุมากกว่า 65 ปี เข้าร่วมประมาณ 1,000 คน และได้เกิดเป็นหนังสือ
``30 Lessons for Living'' โดยใจความที่สำคัญของหนังสือเล่มนี้คือเรื่องของความรักและความสัมพันธ์
ซึ่งใจความนี้ก็สอดคล้องกับการศึกษา ``The Study of Adult Development'' ของ Harvard University
ซึ่งพบว่ากุญแจในการมีความสุขใจชีวิตค่ือ การมีความสัมพันธ์ที่ดี --- Good Relationships

\rule{0em}{1ex}

% \begin{center}
\section*{ข้อสรุปในการได้มาซึ่งความสัมพันธ์ที่ดี}
% \end{center}

% \rule{0em}{1ex}

\subsection*{เลือกคนเคียงคู่ที่มีความใกล้เคียงกับเราและเป็นเพื่อนกับเราได้}
เราจะมีโอกาสที่จะเข้ากับคู่ครองได้มากขึ้น หากคน ๆ นั้นมีความใกล้เคียงกันกับเราในด้านของพื้นฐานครอบครัว
การถูกเลี้ยงดู หรือความเชื่อ แต่ก็ไม่ได้หมายความว่าคนที่ต่างกันกับเราจะเข้ากับเราไม่ได้ เพียงแต่อาจจะเข้ากัน
ได้ยากขึ้นเท่านั้นเองเนื่องจากจะต้องมีการปรับตัวเข้าหากัน อีกทั้งความรักแบบหนุ่มสาวจะค่อย ๆ ลดลงตามกาลเวลา 
คนที่เป็นเพื่อนกับเราได้จึงเป็นคุณสมบัติที่ดี ที่สำคัญคือ อย่าคาดหวังว่าเราจะเปลี่ยนแปลงใครได้

\subsection*{อย่าเสียความสัมพันธ์ไปกับกับดักเล็ก ๆ และ Don't go to bed angry.}
มองอนาคตไกล ๆ อย่าทำให้การทะเลาะกันในเรื่องเล็ก ๆ มาทำลายความสัมพันธ์ที่ดีระหว่างกัน
หากมีเรื่องอะไร ให้คุยกันให้เสร็จ อย่าไปนอนทั้ง ๆ ที่เรื่องราวยังไม่จบ เพราะจะส่งผลเสียต่อสุขภาพกายและใจ
\end{document}