\documentclass{article}
\usepackage{iftex}

\ifxetex
    \usepackage{fontspec}
\else
    \usepackage{luatextra}
\fi
\defaultfontfeatures{Ligatures=TeX}
\usepackage{polyglossia}

\usepackage[autostyle=true]{csquotes}
\setdefaultlanguage{english}
\setotherlanguages{thai}

\setmainfont{CMU Serif}
\setsansfont{CMU Sans Serif}
\setmonofont{CMU Typewriter Text}
For XƎLATEX you have to be a bit more explicit:
\setmainfont{cmun}[
   Extension=.otf,UprightFont=*rm,ItalicFont=*ti,
   BoldFont=*bx,BoldItalicFont=*bi,
 ]
 \setsansfont{cmun}[
   Extension=.otf,UprightFont=*ss,ItalicFont=*si,
   BoldFont=*sx,BoldItalicFont=*so,
 ]
 \setmonofont{cmun}[
   Extension=.otf,UprightFont=*btl,ItalicFont=*bto,
   BoldFont=*tb,BoldItalicFont=*tx,
]

\begin{document}
\section*{Three Areas Configuration}

When you write documents in languages other than English, there are three areas where LATEX has to be configured appropriately:

1. All automatically generated text strings (TOC, List of Figures, ...) have to be adapted to the new language.

2. \LaTeX{} needs to know the hyphenation rules for the current language.

3. Language specific typographic rules. In French for example, there is a mandatory space before each colon character (:).


% \fbox{\textthai{สวัสดี}}

\textrussian{Правда} is
a russian newspaper.
\textgreek{ἀλήθεια} is truth
or disclosure in philosophy


\end{document}