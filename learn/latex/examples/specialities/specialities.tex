\documentclass[a4paper]{article}

\usepackage{blindtext}
\usepackage[utf8]{inputenc}
\usepackage[T1]{fontenc}
\usepackage{makeidx}
\usepackage{fancyhdr}

\title{Specialities}
\author{naiithink}
\date{}

\newcommand{\ex}[2][YYY]{Mandatory arg: #2;
                         Optional arg: #1.}

% \pagestyle{fancy}
% \fancyhf{}
% \fancyhead[L]{\sectionmark}

\setlength{\parindent}{0em}

\makeindex

\begin{document}
\begin{titlepage}
\maketitle
\thispagestyle{empty}
\tableofcontents
\end{titlepage}

\newpage
\pagenumbering{arabic}
\section{Bibliography}
Produce a bibliography with the \verb|thebibliography| \index{Bibliography!thebibliography} environment.

Each entry starts with

\rule{0em}{2ex}

\fbox{\texttt{\textbackslash{}bibitem[item]\{marker\}}}

\rule{0em}{2ex}

The \texttt{marker} is then used to cite the book, article, or paper within the document.

\rule{0em}{2ex}

\fbox{\texttt{\textbackslash{}cite\{marker\}}}

\rule{0em}{2ex}

If the \verb|label| option is omitted, the entries will get enumerated automatically.

\rule{0em}{2ex}

\verb|\begin{thebibliography}{arg}| takes a numerical argument:\\
the widest label expected in the list. This help aligning the \mbox{bibliography} list.

% Partl~\cite{pa} has
% proposed that \ldots
% \begin{thebibliography}{99}
% \bibitem{pa} H.~Partl:
% \emph{German \TeX},
% TUGboat Volume~9, Issue~1 (1988)
% \end{thebibliography}

\newpage
\section{Indexing}

\begin{quote}
Indexes | Indices, which one would you choose?
\end{quote}

\subsection{The \texttt{makeidx} Package}

\rule{0em}{2ex}

\fbox{\texttt{\textbackslash{}usepackage\{makeidx\}}}

\rule{0em}{2ex}

Then put \fbox{\texttt{\textbackslash{}makeindex}} in the preamble.

\rule{0em}{2ex}

\rule{\textwidth}{0.5pt}

\rule{0em}{2ex}

The content of the index specified with the command:

\rule{0em}{2ex}

\fbox{\texttt{\textbackslash{}index\{<key>@<formatted entry>\}}}

\rule{0em}{2ex}

\begin{tabular}{l @{~~~~~~~~} l}
\verb|key|                  &       is for sorting.\\
\verb|formatted entry|      &       \parbox[t]{8cm}{%
will appear in the index.\\
If it is missing, the \texttt{key} will be used.}
\end{tabular}

\rule{0em}{2ex}

When the input file is processed with \LaTeX{},
each \verb|\index| command writes an appropriate index entry,
together with the current page number to a file with the \verb|.idx| extension.

\rule{0em}{2ex}

The \verb|.idx| file can be processed with:

\rule{0em}{2ex}

\fbox{\texttt{makeindex <filename>}}

\rule{0em}{2ex}

This program generates a sorted index with the same base file name,\\
but with the \mbox{extension} \verb|.ind|.

\rule{0em}{2ex}

\begin{center}
\fbox{\texttt{.tex} $\rightarrow$ \texttt{.idx} $\rightarrow$ \texttt{.ind}}
\end{center}

\subsubsection*{Steps}
1. pdflatex file.tex\\
2. makeindex file.ind\\
3. pdflatex file.tex\\
4. pdflatex file.tex

\rule{0em}{2ex}

If now the input file is processed again,\\
this sorted index gets included into the document at \verb|\printindex|.

\rule{0em}{2ex}

\fbox{\texttt{\textbackslash{}printindex}}

\newpage
\subsection{The \texttt{showidx} Package}
The \verb|showidx| package that comes with \LaTeXe{} prints out all index entries
in the left margin of the text. This quite useful for proofreading a document and verify the index.

\rule{0em}{0.5pt}

\fbox{Note that \texttt{\textbackslash{}index} command can affect the layout if not used carefully.}

\newpage
\section{Fancy Headers}
The \verb|fancyhdr| package provides a few simple commands that used to customize header and footer lines.

\subsection*{Six Deprecated Commands}
\subsubsection*{Syntax}
\fbox{\texttt{\textbackslash{}lhead[<even output>]\{<odd output>\}}}

\rule{0em}{2ex}

\begin{table}[!ht]
{\renewcommand{\arraystretch}{1.5}
\renewcommand{\tabcolsep}{6pt}
\begin{center}
\caption{Six Deprecated Commands}
\begin{tabular}[c]{lcr}
\hline
\texttt{\textbackslash{}rhead}  &   \texttt{\textbackslash{}chead}  &   \texttt{\textbackslash{}lhead}\\
\hline
\texttt{\textbackslash{}rfoot}  &   \texttt{\textbackslash{}cfoot}  &   \texttt{\textbackslash{}lfoot}\\
\hline
\end{tabular}
\end{center}}
\end{table}

\subsubsection*{Use These Instead}

\rule{0em}{2ex}

\fbox{\texttt{\textbackslash{}fancyhead[selectors]\{output\}}} or \verb|\fancyfoot|

\subsubsection*{Selectors}
\begin{tabular}{c @{~~~~~~~~} l}
\verb|E|    &   even page       \\
\verb|O|    &   odd page        \\
\verb|L|    &   left side       \\
\verb|C|    &   centered        \\
\verb|R|    &   right side
\end{tabular}

\rule{0em}{2ex}

e.g., \verb|CE,RO|: center of the even pages and to the right side of the off pages.

\rule{0em}{2ex}

To start from scratch, this command will delete the current header/footer configuration.

\rule{0em}{2ex}

\fbox{\texttt{\textbackslash{}fancyhf\{\}}}

\rule{0em}{2ex}

\verb|\fancyhf| is just a merge of \verb|\fancyhead| and \verb|\fancyfoot|.

\rule{0em}{2ex}

% \newcommand{\ex}[2][YYY]{Mandatory arg: #2;
%                          Optional arg: #1.}

\ex

\printindex
\end{document}