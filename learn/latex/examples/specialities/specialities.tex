\documentclass[a4paper]{article}

\usepackage{blindtext}
\usepackage[utf8]{inputenc}
\usepackage[T1]{fontenc}
\usepackage{makeidx}

\title{Specialities}
\author{naiithink}
\date{}

\setlength{\parindent}{0em}

\begin{document}
\begin{titlepage}
\maketitle
\thispagestyle{empty}
\tableofcontents
\end{titlepage}

\newpage
\pagenumbering{arabic}
\section{Bibliography}
Produce a bibliography with the \verb|thebibliography| environment.

Each entry starts with

\rule{0em}{2ex}

\fbox{\texttt{\textbackslash{}bibitem[item]\{marker\}}}

\rule{0em}{2ex}

The \texttt{marker} is then used to cite the book, article, or paper within the document.

\rule{0em}{2ex}

\fbox{\texttt{\textbackslash{}cite\{marker\}}}

\rule{0em}{2ex}

If the \verb|label| option is omitted, the entries will get enumerated automatically.

\rule{0em}{2ex}

\verb|\begin{thebibliography}{arg}| takes a numerical argument:\\
the widest label expected in the list. This help aligning the \mbox{bibliography} list.

% Partl~\cite{pa} has
% proposed that \ldots
% \begin{thebibliography}{99}
% \bibitem{pa} H.~Partl:
% \emph{German \TeX},
% TUGboat Volume~9, Issue~1 (1988)
% \end{thebibliography}

\newpage
\section{Indexing}

\begin{quote}
Indexes | Indices, which one would you choose?
\end{quote}

\subsection{The \texttt{makeidx} Package}

\rule{0em}{2ex}

\fbox{\texttt{\textbackslash{}usepackage\{makeidx\}}}

\rule{0em}{2ex}

Then put \fbox{\texttt{\textbackslash{}makeindex}} in the preamble.

\rule{0em}{2ex}

\rule{\textwidth}{0.5pt}

\rule{0em}{2ex}

The content of the index specified with the command:

\rule{0em}{2ex}

\fbox{\texttt{\textbackslash{}index\{<key>@<formatted entry>\}}}

\rule{0em}{2ex}

\begin{tabular}{l @{~~~~~~~~} l}
\verb|key|                  &       is for sorting.\\
\verb|formatted entry|      &       \parbox[t]{8cm}{%
will appear in the index.\\
If it is missing, the \texttt{key} will be used.}
\end{tabular}

\rule{0em}{2ex}

When the input file is processed with \LaTeX{},
each \verb|\index| command writes an appropriate index entry,
together with the current page number to a file with the \verb|.idx| extension.

\rule{0em}{2ex}

The \verb|.idx| file can be processed with:

\rule{0em}{2ex}

\fbox{\texttt{makeindex <filename>}}

\rule{0em}{2ex}

This program generates a sorted index with the same base file name,\\
but with the \mbox{extension} \verb|.ind|.

\rule{0em}{2ex}

\begin{center}
\fbox{\texttt{.tex} $\rightarrow$ \texttt{.idx} $\rightarrow$ \texttt{.ind}}
\end{center}

\rule{0em}{2ex}

If now the input file is processed again,\\
this sorted index gets included into the document at \verb|\printindex|.

\rule{0em}{2ex}

\fbox{\texttt{\textbackslash{}printindex}}

\newpage
\subsection{The \texttt{showidx} Package}
The \verb|showidx| package that comes with \LaTeXe{} prints out all index entries
in the left margin of the text. This quite useful for proofreading a document and verify the index.

\rule{0em}{0.5pt}

\fbox{Note that \texttt{\textbackslash{}index} command can affect the layout if not used carefully.}


\end{document}