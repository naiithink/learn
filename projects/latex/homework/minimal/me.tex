% !TEX program = xelatex
% -*- coding: utf-8; mode: tex-mode -*-
\documentclass{article}

\usepackage{naiiLayout}
\usepackage{naiiLocale}
\usepackage{blindtext}
\usepackage{verbatim}

% \Title{สวัสดี}
% \author{}
% \date{}

\begin{document}
\flushleft

\begin{Huge}
\rule{0em}{4ex}\\
\noindent\textbf{%
Hello World}
\vspace{3ex}
\end{Huge}

ในสมัยนี้ เส้นแบ่งระหว่างวัยกลางคนกับวัยสูงอายุเริ่มจางลง สังเกตได้จากคนที่อายุ 60--70 ปีที่ดูภายนอกแล้วเหมือน
กับว่าอายุเพียงแค่ 40--50 ปี หรือที่เราเรียกว่า \mbox{``Young Old''} ซึ่งสิ่งหนึ่งที่ทำให้เส้นแบ่งนี้จางลงได้คือ องค์ความรู้
ทางการแพทย์ที่ก้าวหน้ามากขึ้น

\rule{0em}{1ex}

Dr. Karl Pillemer อาจารย์จาก Cornell University ได้ทำการศึกษาเกี่ยวกับเคล็ดลับในการมีความสุขกับชีวิต
ในมุมมองของผู้สูงอายุ โดยอาจารย์ฯ ได้เริ่มการศึกษาโดยให้ผู้สูงอายุจำนวนประมาณ 100 คน ได้เขียนเล่าแบบ
free-style ทำนองว่าอะไรคือกุญแจสำคัญของความสุข และหากย้อนเวลากลับไปได้อยากจะบอกอะไรบ้าง
หลังจากนั้น อาจารย์ฯ ก็ได้ทำแบบสอบถามส่งให้กับผู้สูงอายุอีกหลายร้อยคน ทำให้ได้ข้อมูลที่น่าสนใจชุดหนึ่ง
ที่บางคำตอบมีมุมมองที่น่าสนใจ อาจารย์ฯ จึงได้เชิญผู้สูงอายุเหล่านั้นมาสัมภาษณ์เชิงลึกเพื่อเจาะลึกถึงข้อมูล

\rule{0em}{1ex}

รวม ๆ แล้ว การศึกษานี้มีผู้ที่อายุมากกว่า 65 ปี เข้าร่วมประมาณ 1,000 คน และได้เกิดเป็นหนังสือ
\mbox{``30 Lessons for Living''} โดยใจความที่สำคัญของหนังสือเล่มนี้คือเรื่องของความรักและความสัมพันธ์
ซึ่งใจความนี้ก็สอดคล้องกับการศึกษา \mbox{``The Study of Adult Development''} ของ Harvard University
ซึ่งพบว่ากุญแจในการมีความสุขใจชีวิตคือ การมีความสัมพันธ์ที่ดี --- Good Relationships

% \verbatiminput{hello.txt}

\begin{verbatim}
#include <stdio.h>

int main(void)
{
    printf("hello, world\n");
    return 0;
}
\end{verbatim}
\end{document}